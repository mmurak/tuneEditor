\chapter{アップデートで追加された機能}

\section{バージョン4.07で追加された機能}

\subsection{テキスト未入力域で行間表記の平行線表示を抑止する}

「FullBaseline=0」と指定すると、テキスト未入力域では行間表記の平行線が表示されないようになります。指定しない、あるいは「0」以外を指定すると、従来どおり画像ファイルの右端まで平行線が描画されます。

\subsection{画面の左端から指定ピクセル数分だけ、平行線の表示を抑止する}

複数人の対話ダイアログを表記するような場合、画面の左端に話者を識別する名称を表記する場合が多いはずです。この場合、そういった名称には行間表記を使用しないため、平行線は不要となるはずです。

こういった用途のために「sf」パラメーターを追加しました。このパラメータ(Speaker Fieldの略です)を指定することにより、画面の左端から指定したピクセル数分だけ、平行線の描画が抑止されます。

なお、表示が抑止されるのは平行線だけであるため、話者の識別名称に音節核候補文字が含まれている場合、対応するドット(群)が表示されます。この場合、[Make Item Invisible]コマンドを使用して、ドットを消し込んでください。

使用に際しては、話者名込みのかたちでダイアログを入力した後、Control+kを押下し、「sf=50」などと入力してみて、話者識別名の下の表示がどの程度抑止されるのかを確認し、適当な数値を探し出してください。
